\documentclass[finalProposal.tex]{subfiles}
\begin{document}
\onehalfspacing

\noindent{\Large Results and Discussion}

\bigskip

By the end of the project, all of the different parts were accomplished to differing degrees of success. The biggest change from the original plan was the switch in microcontrollers from the Raspberry Pi over to the Arduino. This change was made due to time constraints and lack of familiarity with the Pi. The very successful components of the project are the Bluetooth communication and the movement and sound data. The components that were finished but have quite a bit of room for improvement are the GSR and heartbeat sensors. The GSR sensor seems to give reasonable data, but more research needs to be put into calibration and the classification of which values are reasonable. The heartbeat sensor was the most troublesome. The homemade sensor did not work at all due to irregularities in the wavelengths of both the emitter and detector. Due to this issue, a heartbeat module was purchased from Sparkfun. This returned data, but the data was very skewed and did not reflect an actual heart beat. Fortunately, some data was able to be collected and displayed for a proof of concept. Looking forward, more research would need to be done on GSR and reliable data, and a regular and reliable heartbeat sensor would need to be developed. Additionally, power concerns as well as wear-ability concerns would need to be addressed. If the entire module needs to fit in a glove, it cannot have huge batteries and most importantly, needs to be comfortable enough to sleep with.

Low power Bluetooth and sleeping between reading data points would be good places to start in looking to improve the viability of this monitor as a product. The module itself currently takes too much power as it is constantly on and receiving Bluetooth data every second. In order to make this a long term solution, the frequency of the data as well as the amount of power that the module needs would need to be cut down substantially. Additionally, an alarm that looks at data trends and wakes the user near when they would like to get up would be a critical feature to implement. Many sleep monitor applications currently do this, and this feature is their biggest selling point.

As it stands, the product performs its task quickly but not necessarily reliably. The data is all there and sent in a timely manner, but the accuracy of the data sent (as discussed before) is lacking especially with the GSR and heartbeat. If the project was to be redone, off the shelf circuits would be considered for the GSR and heartbeat sensors. Though making these circuits was interesting and challenging, it didn't contribute to the core of the project, and ultimately held the project back by a few weeks. Given more time and more off the shelf components, a different microcontroller like the Raspberry Pi could be utilized.

The total cost of the project was \$78. The Arduino, heartbeat sensor, and Bluetooth module all cost \$25 and the GSR circuit components cost about \$3.


\end{document}
