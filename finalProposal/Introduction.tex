\documentclass[finalProposal.tex]{subfiles}
\begin{document}
\onehalfspacing

\noindent{\Large Introduction}

\bigskip

At one point or another throughout their lives, most adults experience some form of sleeping issue. These issues are often frustrating and can be detrimental to happiness, productivity, and quality of life. According to the University of Maryland Medical Center, 40 million adults in America suffer from chronic sleep disorders and a large portion of these cases are unidentified and undiagnosed. 
Unfortunately, there are not many good solutions for an average person to characterize their sleep cycles and identify issues. Sleep studies performed by medical professionals are prohibitively expensive, and other solutions such as current phone apps do not take enough data to give a comprehensive result.

The sleep sensor provides a solution between expensive medical testing and lackluster phone applications by taking components of both of these solutions. The sleep sensor uses an android app as its main interface with the end user. This allows the user to interact with their sleep data without buying an expensive stand alone tool. Additionally, the use of android apps allows easy integration of sound and accelerometer data. Most android sleep applications stop with noise and motion data, this sleep sensor sets itself apart by having an additional module that tracks extra data. This module, which communicates over Bluetooth, tracks heart rate and galvanic skin response (conductance of the skin which shifts throughout sleep cycles) by taking data from a glove placed on the hand before sleep. 

The glove prototype as well as the analog circuits contained within it (attached to a Raspberry Pi microcontroler for prototyping) will be ready by the 20$^{th}$ of November, and the Application and communication will be ready by the 4$^{th}$ of December. The remaining time will be dedicated to the final write up as well as any enhancements deemed necessary. 

This prototype can be a stepping stone to future development. Making the device smaller more powerful, and more convenient is the first step to creating a product that can help adults identify issues with their sleep. Though the device may never be able to diagnose disorders, it can give clear indications that something is wrong. 
 
\end{document}
