\documentclass[finalProposal.tex]{subfiles}
\begin{document}
\onehalfspacing

\noindent{\Large Design and Testing Methodology}

\bigskip

Because the time frame of the final project is limited, design choices in both software and hardware were centered around ease of use and availability of documentation and online resources. 
The first decision was choosing the embedded system to use. The three choices were the ARM M0 Cortex, the Arduino, and the Raspberry Pi. Though the M0 cortex was the easiest solution as they are provided by the course, the Raspberry Pi was chosen. The Pi has an advantage over the M0 in terms of online resources and documentation, and an advantage over the Arduino in terms of flexibility and an operating system. The next challenge was choosing which sensors needed to be included and how they should be incorporated. The accelerometer and microphone were an easy choice because they are included on every Android device and didn't necessitate extra hardware. The other two sensors, heart rate monitor and GSR monitor, were chosen because they give the most insight into the condition of the person sleeping, while being discrete. Other sensors, such as brain activity monitors may give useful information, but it is difficult to incorporate into a design that needs to be discrete. Probes attached to a users forehead may have their place, but it is certainly not something that needs to be done every night to monitor sleep. The heart rate sensor and GSR monitor will both be built as analog circuits because they are both relatively easy and inexpensive compared to their off the shelf counterparts. The final design choice at this point in the project was the mode of communication. The options were hardware USB, Bluetooth, and Wifi. Bluetooth was chosen because of the relative ease of the interface, and experience already gained in class. USB was ruled out because it would make the device difficult to sleep with, and Wifi was ruled out because Bluetooth had the edge on ease of use.

Throughout development, we will implement tests to ensure that our data, especially from our analog circuits is reasonable and accurate. In order to test the data from our heart rate sensor, we will take baseline pulses, and using an oscilloscope make sure that the peaks in the data match the pulse frequency. Additionally we will check that the data is still correct after going through the ADC by comparing the peaks in the data to the ones from the oscilloscope. The GSR monitor is slightly harder to test as its results are more ambiguous. We will begin testing by making a simple lie detector. When someone lies their Galvanic Skin Response changes as they sweat more. This lie detector will be set up to turn on an LED when someone is lying. Once we get reasonable results from the lie detector, we can change that circuit to feed into the ADC and monitor those values. The same lying test will be performed except that the flashing LED will be replaced by changes in voltage. Once both of the sensors are relaying reasonable data, we can test the communication between the Raspberry Pi and the Android device. This can be accomplished in two stages, the first being a forced response generated from waveform generators, this can be run for several hours to make sure that the data is being accurately transmitted and that no samples are being missed. The next state is to test it during a sleep session in order to make sure that the data is indeed variable and that the change is noticeable. Once we are sure all of the different parts are working, we will use the device while sleeping for several days to see the data and make sure it lines up with an average nights sleep (i.e. falling asleep, REM cycles, and waking period). This data can then compared to the graph displayed to the end user to find any differences or discontinuities. 

 
\end{document}
