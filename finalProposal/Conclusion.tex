\documentclass[finalProposal.tex]{subfiles}
\begin{document}
\onehalfspacing

\noindent{\Large Conclusion}

\bigskip

Though there were setbacks, the project was ultimately a success. A sleep monitor was developed in accordance with the project specifications. This sleep monitor, controlled by an android application was able to collect and display 4 different data points. It collected movement and sound data from the android device itself, and received Bluetooth messages about the GSR value and the heartbeat. It was able to store this data and display the results for an entire nights worth of data. The homemade GSR circuit retrieved somewhat reasonable data that would show general trends over time. The heartbeat sensor from Sparkfun was able to give some good data but was very particular about where it was placed and how much pressure there was on it. Though there were some difficulties it did read heartbeat data. Both of these peripheral sensors fed data into the Arduino which then relayed it over Bluetooth to the paired Android device. At the end of the night the android was able to display each set of data separately on a large graph. Overall, the monitor performed well, but as discussed in the results there are a lot of improvements to be made. Overall, this monitor is an excellent proof of concept for what can be accomplished.

The project took about 25 hours to complete. Which is a pretty reasonable amount of time for a final project. The class overall was very good, the only thing that I can think to change is giving students a little bit more help on some of the tougher labs. Most labs were fine, but one or two felt like we were left out to dry as the TAs didn't know what do. Those instances were frustrating but the class was still an overall great experience.

\bigskip

\noindent Top Three Projects
\begin{enumerate}
\item Alex's quadcopter control system
\item The android controlled jig
\item The homemade cell phone
\end{enumerate} 

\end{document}
